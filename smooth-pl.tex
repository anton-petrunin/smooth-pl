\documentclass[a4paper,10pt]{amsart}
\usepackage{smooth-pl}
\hypersetup{pdftitle={Smoothing 3D polyhedral spaces}%
,pdfauthor={Nina Lebedeva, Vladimir Matveev, Anton Petrunin and  Vsevolod Shevchishin}
}

\begin{document}
%\pagestyle{empty}

\title{Smoothing 3-dimensional polyhedral spaces}

\author[Lebedeva]{Nina Lebedeva}
\address{N. Lebedeva\newline\vskip-4mm
Steklov Institute, St. Petersburg, Russia.
\newline\vskip-4mm
Mathematical Department of
St. Petersburg State University, Russia.}
\email{lebed@pdmi.ras.ru}

\author[Matveev]{Vladimir Matveev}
\address{V. Matveev\newline\vskip-4mm
Institut f\"ur Mathematik, Friedrich-Schiller-Universit\"at Jena,  Germany.
}
\email{vladimir.matveev@uni-jena.de}

\author[Petrunin]{\\Anton Petrunin}
\address{A. Petrunin\newline\vskip-4mm
Mathematical Department, Pennsylvania State University, USA}
\email{petrunin@math.psu.edu}

\author[Shevchishin]{Vsevolod Shevchishin}
\address{V. Shevchishin\newline\vskip-4mm
National Research University, Higher School of Economics, Moscow, Russia.
}
\email{shevchishin@gmail.com}

\thanks{N.~Lebedeva was partially supported by RFBR grant 
14-01-00062.}
\thanks{A.~Petrunin was partially supported by NSF grant DMS 1309340.}
\date{}
\begin{abstract}
We show that 3-dimensional polyhedral manifolds 
with nonnegative curvature in the sense of Alexandrov
can be approximated by nonnegatively curved 3-dimensional Riemannian manifolds.
\end{abstract}
\maketitle

\section{Introduction}

We define \emph{a polyhedral space} as a complete metric space that admits a locally finite triangulation 
such that each simplex is isometric to a simplex in a Euclidean space.
If in addition, the space is homeomorphic to a manifold,
we call it \emph{a polyhedral manifold}.

\begin{thm}{Main Theorem}\label{thm:main}
Assume $P$ is a compact 3-dimensional polyhedral manifold with nonnegative curvature in the sense of Alexandrov.
Then there is a 
Ricci flow $L^t$ on a 3-dimensional manifold
defined in the time interval $(0,T)$
such that $L^t\z\to P$ as $t\to0$ in the sense of Gromov--Hausdorff
and sectional curvature of $L^t$ is nonnegative for any $t$.

In particular, $P$ is a Gromov--Hausdorff limit 
of 3-dimensional nonnegatively curved Riemannian manifolds. 
\end{thm}

\parit{Sketch.}
We introduce so-called \emph{$K[\eps]$-pinching}.
A 3-dimensional Riemannian manifold $M$ satisfies $K[\eps]$-pinching
if at any point $x\in M$ and any sectional direction $\sigma_x$ at $x$ we have
\[\Sec(\sigma_x)+\tfrac\eps2\cdot\R(x)\ge 0;\]
here $\Sec(\sigma_x)$ denotes sectional curvature in the sectional direction $\sigma_x$ and $\R(x)$ denotes the scalar curvature at $x$.

In Proposition~\ref{prop:smooth},
we construct a sequence of 3-dimensional Riemannian manifolds $M_n$ converging to $P$ such that
 $M_n$ is  $K[\tfrac1n]$-pinched for each $n$.
 
Further, 
we consider the Ricci flows $M_n^t$ with the initial data $M_n^0=M_n$.
Passing to a limit $L^t$ of $M_n^t$ as $n\to\infty$ we obtain a Ricci flow which is $K[0]$-pinched; that is, it has nonnegative sectional curvature. \qeds

After this work was done, we learned that a similar technique was used by Wolfgang Spindeler \cite{spindeler} to smooth 3-dimensional Alexandrov spaces of a certain type.

\parbf{Acknowledgments.}
We would like to thank 
Yuri Burago,
Bruce Kleiner,
Thomas Richard,
Miles Simon,
and Burkhard Wilking.
A big part of the paper was written during our stay at Istanbul Center for Mathematical Sciences,
we want to thank this institute for its excellent working conditions.


\section{Remarks and motivations}

\parit{Main motivation.}
This paper was motivated by the following conjecture. 

\begin{thm}{Smoothing conjecture}\label{conj:main}
Assume $P$ is a compact polyhedral space with nonnegative curvature in the sense of Alexandrov. 
Then $P$ can be presented as a limit of Riemannian  orbifolds\footnote{More formally \emph{as a limit of underlying metric spaces of Riemannian orbifolds}.}
with \emph{nonnegative cosectional curvature}.
\end{thm}

We are about to explain the meaning of \emph{nonnegative cosectional curvature}.
Before that let us state some of its features.
\begin{enumerate}[(i)]
\item In the 3-dimensional case 
\emph{nonnegative cosectional curvature} 
has the same meaning as 
\emph{nonnegative sectional curvature}.
\item In the 4-dimensional case 
\emph{nonnegative cosectional curvature} 
has the same meaning as 
\emph{nonnegative curvature operator}.
\item In dimensions $5$ and higher 
any metric with
\emph{nonnegative cosectional curvature} 
also has 
\emph{nonnegative curvature operator},
but the converse does not hold.
\end{enumerate}

The curvature tensors on tangent space $T$
form a subspace of $\mathrm{S}^2(\Lambda^2(T))$ which will be denoted by $\mathrm{A}^4(T)$;
here $\mathrm{S}^2(\Lambda^2(T))$ denotes the symmetric square of the space of bivectors of $T$.
Tensor $R\in \mathrm{S}^2(\Lambda^2(T)$ is a curvature tensor 
if and only if it can be presented as a finite sum
\[R=\sum_i\lambda_i\cdot(x_i\wedge y_i)^2,\eqno({*})\]
where $x_i, y_i\in T$ and $\lambda_i\in\RR$;
the latter is equivalent to the 3-cyclic sum identity
on~$R$. 

We say that a curvature tensor $R$ has \emph{nonnegative cosectional curvature} if we can find a presentation $({*})$ 
such that $\lambda_i\ge 0$ for all $i$.
We say that $R$ has \emph{positive cosectional curvature} if it can be presented as a sum of the curvature tensor of a round sphere and a curvature tensor  with \emph{nonnegative cosectional curvature}.

Note that the action of $\mathrm{GL}(T)$ extends to the action on $\mathrm{A}^4(T)$.
It turns out that the set of \emph{nonnegative cosectional curvature} forms the minimal closed convex $\mathrm{GL}(T)$-invariant cone which contains the curvature tensor of the unit sphere.
By a surprising coincidence, the largest proper cone with this property is formed by curvature tensors with nonnegative sectional curvature.

Note that the scalar product on $T$ extends to the scalar product on $\mathrm{A}^4(T)$.
It turns out that the curvature tensor $R\in \mathrm{A}^4(T)$ has nonnegative cosectional curvature if and only if
\[\langle R,S\rangle\ge 0\]
for any tensor $S\in \mathrm{A}^4(T)$
with nonnegative sectional curvature.
This property justifies the term \emph{cosectional}.

The smoothing conjecture (\ref{conj:main}) was motivated by the following theorem from~\cite{petrunin}.

\begin{thm}{Theorem}
If a Riemannian manifold $M$ admits a Lipschitz approximation by  polyhedral spaces
with nonnegative curvature in the sense of Alexandrov,
then $M$ has nonnegative cosectional curvature.

Conversely, 
if $M$ is a compact $m$-dimensional Riemannian manifold with positive cosectional curvature
then it admits a Lipschitz approximation by $m$-dimensional polyhedral spaces
with nonnegative curvature.
\end{thm}

Note that Theorem~\ref{thm:main} proves the smoothing conjecture for 3-dimensional polyhedral manifolds. 

In fact the proof of Theorem~\ref{thm:main} can be modified to give a complete proof of the smoothing conjecture in the 3-dimensional case.
To do this note that any $3$-dimensional polyhedral $P$
is isometric as a quotient $\bar P/\iota$, where $\bar P$ is a polyhedral manifold and $\iota\:P\to P$ is an isometric involution.
It remains to check that all the constructions in the proof can be made invariant with respect to a given isometric involution.

If a polyhedral space with nonnegative curvature 
admits a smoothing, then the link of each simplex has to be homeomorphic to a sphere.
The later was proved by Vitali Kapovitch in \cite[Theorem 1.3]{kapovitch}.
It is expected that this is also a sufficient condition.
Note that not some polyhedral manifolds do not have this property.
For example, let $K$ be the cone over spherical suspension over the Poincar\'e homology sphere which can be also thought as a
quotient of $\RR^5=\RR^4\times\RR$ by
the binary icosahedral group acting on the $\RR^4$-factor.
The space $K$ is a topological manifold, but it has edges  with links homeomorphic to the Poincar\'e homology sphere.
In particular, $K$ can not be approximated by smooth manifolds with a lower curvature bound.

Ut was noted by Christoph B\"ohm and Burkhard Wilking \cite{boehm-wilking} that
nonnegative cosectional curvature survives under Ricci flow.
It gives hope that smoothing of a polyhedral space 
using Ricci flow used in our proof
can also work in higher dimensions.
Absence of analog of Simon's theorem (see Corollary~\ref{cor:simon}) seems to be the main obstacle in this way.
In particular, we do not know the answer to the following question.

\begin{thm}{Question} 
Assume $M$ is a compact $m$-dimensional Riemannian manifold,
$\diam M=D$, $\vol M=v_0$,
and the curvature operator (or cosectional curvature) of $M$ is at least $\kappa$.
Consider the Ricci flow $M^t$ with the initial data $M^0=M$
defined in the maximal interval $[0,T)$.

Is there a positive lower bound for $T$ in terms of $m$, $D$, $\kappa$, and $v_0$? 
\end{thm}


\parit{Smoothing 3-dimensional Alexandrov spaces.}
The following problem goes back to the end of the 80-s.
Our paper gives a partial answer.
Another partial answer is given by Wolfgang Spindeler \cite{spindeler}.
A more general problem of that type was considered by Thomas Richard \cite{richard}.

\begin{thm}{Open problem}
Prove that any compact $3$-dimensional Alexandrov space that is homeomorphic to a manifold admits an approximation by $3$-dimensional Riemannian manifolds with the same lower curvature bound.
\end{thm}



\section{Preliminaries and notations}



\parbf{Alexandrov's embedding theorem.}
The Alexandrov embedding theorem states in particular that 
\textit{any Riemannian metric with curvature $\ge 1$ on the $2$-sphere
is isometric to a convex surface in the unit $3$-sphere.}
Applying the cone construction 
to the source 
and target spaces 
of this embedding we obtain the following.

\begin{thm}{Corollary}\label{cor:alex}
Let $K$ be a Euclidean cone with nonnegative curvature in the sense of Alexandrov which is homeomorphic to $\RR^3$.
Then $K$ is isometric to the surface of a convex cone in $\RR^4$.

Moreover, if $K$ is smooth away from its tip 
then the surface is smooth away from the tip.
\end{thm}

\parbf{Hamilton's convergence.}
The following statement follows from the main theorem in \cite{hamilton-compactness} and 
the estimate on injectivity radius in terms of diameter, volume, dimension, and upper curvature bound.

\begin{thm}{Proposition}\label{prop:ricc-convergence}
Let $M_n^t=(M,g_n^t)$ be a sequence of $m$-dimensional Ricci flows on a compact manifold $M$
defined in the fixed interval $t\in(0,T)$. 
Assume the following two conditions:
\begin{enumerate}[(a)]
\item for each compact subinterval $\II \subset (0,T)$, 
the curvature of $M_n^t$ is uniformly bounded for all $t\in \II$;
\item there are real numbers $v_0>0$ and $D$ such that 
\[\vol M_n^t\ge v_0\ \ \text{and}\ \ \diam M_n^t\le D\] 
for any $t$ and $n$.
\end{enumerate}
Then, after passing to a subsequence, the solutions converge smoothly to a complete Ricci flow solution $M_\infty^t$, defined for all $t\in (0,T)$.
\end{thm}

\parbf{Simon's theorem.}
The following statement from the theorem of Miles Simon \cite[Theorem 1.9]{simon-non-collapsed}.

\begin{thm}{Corollary}\label{cor:simon}
Given positive reals $v_0$, $D$ and $\kappa$
there are positive real constants $K$ and $T_0$
such that the following condition holds.

Suppose $M^0=(M,g^0)$ is a compact $3$-dimensional manifold 
such that 
\begin{align*}
\Sec_{M^0}&\ge -\kappa,
&
\vol M^0&\ge v_0,
&
\diam M^0&\le D.
\end{align*}
Let $M^t=(M,g^t)$ be the solution of Ricci flow with initial data $M^0$.
Then $M^t$ is defined in $[0,T_0)$ and 
\begin{align*}
\Sec_M&\ge -K,
&
\vol M^t&>\tfrac{v_0}2,
&
|\Rm_{g^t}|\le \frac Kt
\end{align*}
for any $t\in [0,T_0)$.
Moreover for any two points $x,y\in M$, we have
$$\bigl||x-y|_{g^t}-|x-y|_{g^s}\bigr|
\le 
K\cdot \sqrt{|s-t|},$$
where $|x-y|_{g^t}$ denotes the distance from $x$ to $y$ 
induced by the metric tensor $g^t$. 
\end{thm}

\parbf{Chen--Xu--Zhang pinching.}
Let $(M,g)$ be a compact 3-dimensional Riemannian manifold.
Fix $\eps\ge0$.
We say that $g$ is \emph{$K[\eps]$-pinched} if 
\[\Sec(\sigma_x)+\tfrac\eps2\cdot\R(x)\ge 0\]
for any tangent sectional direction $\sigma_x$ at any $x\in M$.
This condition defines a convex $O(3)$-invariant cone in the space of curvature tensors $\mathrm{A}^4(\RR^3)$.

\begin{thm}{Lemma}\label{lem:pinching}
Let $\eps\ge 0$ and 
$M^t$ be a solution of Ricci flow defined in the interval $[0,T)$.
Assume $M^0$ has $K[\eps]$-pinched curvature,
then so is $M^t$ for any $t\in[0,T)$.
\end{thm}

The lemma above is a partial case of the main theorem of 
Bing-Long Chen,
Guoyi Xu, and
Zhuhong Zhang \cite{chen-xu-zhang}.
A slightly weaker statement was proved by Miles Simon \cite[Lemma 5.1]{simon},
and it can be used in our proof the same way.


\section{The proofs}\label{sec:rough-smooth}

The proof of our main theorem will be given in the very end of this section;
it is based on the following proposition.
By $\dist_\GH$ we will denote the Gromov--Hausdorff metric.

\begin{thm}{Proposition}\label{prop:smooth}
Assume $P$ is a compact 3-dimensional polyhedral manifold.
Then there is a real value $\kappa$ and a sequence of 3-dimensional Riemannian manifolds $M_1, M_2,\dots$ 
such that  $M_n$ is $K[\tfrac1n]$-pinched,  $\dist_\GH(M_n,P)<\tfrac1n$ and $\Sec_{M_n}\ge \kappa$ for each $n$.
\end{thm}
 

Before coming to the proof, we need to discuss the structure of singularities of $3$-dimensional polyhedral manifolds.

Let $P$ be as in the proposition.
Assume $x\in P$ is a \emph{singular point};
that is, $x$ does not have a neighborhood that is isometric to an open subset of $\RR^3$.

The point $x$  will be called an \emph{essential vertex} if the cone at $x$ 
is not isometric to the product $K\times \RR$ where $K$ is a two-dimensional cone.
Note that the set of essential vertices consists of isolated points in $P$ and is therefore finite.

The remaining singular points form a finite number of connected components.
The closure of each component will be called an \emph{essential edge}
and the points in the corresponding connected component
will be called the \emph{interior points} of this edge.

Consider an essential edge $E$.
Note that a neighborhood $U$ of any interior point of $E$ is isometric to an open subset $U'$ in $K\times \RR$ for some two-dimensional cone $K$.
Under this isometry $\iota\: U\to U'$,
the points on $E$,
are sent to $o_K\times \RR$ in $K\times \RR$,
where $o_K$ denotes the tip of $K$.

Assume $\tau$ is a triangulation of $P$.
Note that any essential vertex of $P$ is a vertex of $\tau$
and any essential edge of $P$ is a union of some edges of $\tau$.
On the other hand, a vertex of $\tau$ might be nonessential,
as well as an edge of $\tau$ may not lie in an essential edge of $P$.

Note that each essential edge is a local geodesic.
The essential edge is called \emph{closed} if the geodesic is periodic;
otherwise, it is called \emph{open}. 
In the latter case, the edge connects two vertices 
or a vertex to itself.

Note that the cone $K$ above can be chosen 
the same for all the interior points on $E$.
Assume $\theta_E$ denotes the total angle around the edge;
it is the total angle of the cone $K$.
Then the value $\omega_E=2\cdot\pi-\theta_E$ will be called the \emph{curvature} of $E$. 

By the definition an essential edge has nonvanishing curvature;
since $P$ has nonnegative curvature,
the curvature of any essential edge is positive.

\parit{Proof of Proposition~\ref{prop:smooth}.}
The construction of the sequence $M_n$ uses two procedures  
(1) \emph{edge smoothing} and 
(2) \emph{vertex smoothing}.

First, we apply \emph{edge smoothing} to $P$.
It produces %??? produce>produces
a sequence of manifolds $M_n'$
with isolated singular points for each vertex of $P$
such that 
\begin{enumerate}[(i)]
\item $\dist_\GH(M_n',P')<\tfrac1{2\cdot n}$ for every $n$;
\item The curvature of $M_n'$ is $K[\tfrac1n]$-pinched at any smooth point;
\item Each singular point in $M_n'$ has a conic neighborhood with nonnegative curvature in the sense of Alexandrov.
\end{enumerate}

\parit{Edge smoothing.} 
Assume $E$ is a closed edge.
Denote by $\ell$ its length.
Note that there is 2-dimensional cone $K$, 
a disk $D\subset K$,
and an isometry $\iota\:D\to D$
such that a tubular neighborhood $U$ of $E$ is locally isometric to the space glued from cylinder $D\times [0,\ell]$
by the map $(p,0)\mapsto(\iota(p),\ell)$.

Let us embed $K$ as the graph \[z=k\cdot\sqrt{x^2+y^2}\]
in $(x,y,z)$-space.
Fix a smooth convex even function $\phi(t)$ such that $\phi(t)=|t|$ if $|t|>1$.
Given $\eps>0$, set $\phi_\eps(t)=\eps\cdot\phi(\tfrac t\eps)$.
Denote by $K'_\eps$ the graph
\[z=k\cdot\phi_\eps(\sqrt{x^2+y^2})\]
with induced length metric.

Assume $\eps$ is sufficiently small.
Then there is a rotationally symmetric disk $D'$ in $K'_\eps$
which is isometric to $D$ near the boundary.
Denote by $\iota'\:D'\to D'$ the isometry which coincides with $\iota$ near the boundary of $D$.


Cut the neighborhood $U$ from $P'$ and glue instead $D'\times [0,\ell]/\sim$,
where $\sim$ is the minimal equivalence relation
such that $(p,0)\sim (\iota'(p),\ell)$ 
for any $p\in D'$.
This way we can smooth all the closed edges.

\smallskip

Now assume $E$ is an open edge in $P$.

Note that there a rotationally symmetric convex cone in $\RR^3$ 
with surface $K$
such that 
$E$ has a neighborhood $\Omega$ 
which is locally isometric to the intersection of a convex neighborhood  of $\{0\}\times(0,\ell)$ in $\RR^3\times\RR=\RR^4$ with $K\times \RR$.

Fix a concave smooth function $f\:[0,\ell]\to\RR$ 
such that for all sufficiently small $t\ge 0$ we have
$f(t)=t$ 
and $f(\ell-t)=t$.

Fix sufficiently small $\eps>0$.
Consider the hypersurface in $\RR^4$ formed by one parameter family of smoothings $K'_{\eps\cdot f(t)}\times\{t\}$;
it also can be described as a graph in $(x,y,z,t)$-space
\[z
=
k\cdot \phi_{\eps\cdot f(t)}
\left(
\sqrt{x^2+y^2}
\right)\]

Let $k_1\le k_2\le k_3$ be the principal curvatures of the obtained surface at a given point.
For the straightforward choice of functions $\phi$ and $f$,
we have that 
(1) $k_1\le 0\le k_2\le k_3$,
(2) $k_1\cdot k_3\ge \kappa$ for some fixed negative constant $\kappa$ and any $\eps>0$
and (3) $\sup_{k_1\ne 0}\tfrac{k_1}{k_2}\to 0$ as $\eps\to 0$.
In particular, assuming that $\eps$ is sufficiently small, 
the constructed patch has $K[\tfrac1n]$-pinched curvature 
and the sectional curvature at least $\kappa$; here the constant $\kappa$ is independent of $n$.

Note that after the edge smoothing 
the ends of the edge 
have conic neighborhoods with nonnegative curvature.

Applying these operations to all edges of $P$ 
for sufficiently small $\eps=\eps_n>0$ we get the sequence $(M_n')$. 

\medskip

The vertex smoothing produces a Riemannian manifold $M_n$ for each $M_n'$;
it only changes $M_n'$ in a small neighborhood of each vertex leaving this neighborhood nonnegatively curved.

\begin{wrapfigure}{o}{65mm}\begin{lpic}[t(-0mm),b(-0mm),r(0mm),l(-0mm)]{pics/hat(1)}\lbl[bl]{27,24;$2{\cdot}\delta$}\end{lpic}\end{wrapfigure}%???

\parit{Vertex smoothing.}
For any singular point,
there is $\eps>0$ such that its $\eps$-neighborhood is conic.
By Corollary~\ref{cor:alex}, this neighborhood
is isometric to an open set in the surface $K$ of convex cone in $\RR^4$.
The surface of cone is smooth at all points except the tip.

We can assume that coordinates $(x,y,z,w)$ in $\RR^4$ are chosen in such a way that $K$ forms a graph $w=f(x,y,z)$ for a nonnegative convex positive-homogeneous function $f$.

Fix a convex function $\phi\:\RR_{\ge0}\to \RR_{\ge0}$ which is constant at the points 
$\delta$-close to $0$ 
and identity $2\cdot\delta$-away from zero.  
Note that the graph of the composition 
\[\phi\circ f\:\RR^3\to\RR\] 
forms a smooth convex hypersurface in $\RR^4$ which coincides with $K$ sufficiently far from zero.

Cut a conic neighborhood for each vertex of $M_n'$
and glue instead the graph of composition obtained above 
for small enough $\delta>0$. 
This operation produces the needed Riemannian manifold $M_n$.
\qeds

\parit{Proof of the main theorem.}
Let $M_n$ be the sequence of manifolds provided by Proposition~\ref{prop:smooth}.

Consider the sequence of Ricci flow solutions $M_n^t$ 
with initial data $M^0_n=M_n$.
Applying Corollary~\ref{cor:simon}, 
we get $M_n^t$ are defined in a fixed time interval $[0,T_0)$.

Applying Corollary~\ref{cor:simon} together with Proposition~\ref{prop:ricc-convergence}, 
we can pass to a subsequence of $M_n^t$ which converges to a solution of Ricci flow $L^t$ for  $t>0$.

Since each $M_n$ is $K[\tfrac1n]$-pinched,
Lemma~\ref{lem:pinching} implies that $M_n^t$ is $K[\tfrac1n]$-pinched for any $t$.
It follows that $L^t$ is $K[0]$-pinched;
that is, $L^t$ has nonnegative sectional curvature for all $t>0$.

By Corollary~\ref{cor:simon} each family $M_n^t$ is uniformly continuous with respect to the Gromov--Hausdorff metric.
Therefore so is the family $L^t$.
In particular, $P$ is the Gromov--Hausdorff limit of $L^t$ as $t\to 0$
since it coincides with the limit of $M_n$ as $n\to \infty$.
\qeds


\begin{thebibliography}{52}
\bibitem{boehm-wilking} B\"ohm, C.; Wilking, B.,
\textit{Manifolds with positive curvature operators are space forms.} Ann. of Math. (2) 167 (2008), no. 3, 1079--1097.

\bibitem{chen-xu-zhang}
Chen, B.;  Xu, G., Zhang, Z.,
\textit{Local pinching estimates in 3-dim Ricci flow},
Math. Res. Lett. Vol. 20 (2013), No. 5, 845--855

\bibitem{hamilton-compactness}
Hamilton, R. S.,
\textit{A Compactness Property for Solutions of the Ricci Flow,}
American Journal of Mathematics
Vol. 117, No. 3 (1995), pp. 545--572

\bibitem{kapovitch} 
Kapovitch, V.
\textit{Regularity of limits of noncollapsing sequences of manifolds.}
Geom. Funct. Anal. 12 (2002), no. 1, 121--137. 

\bibitem{petrunin} Petrunin, A., 
\textit{Polyhedral approximations of Riemannian manifolds.}
Turkish Journal of Mathematics 27 (1), 173--188.

\bibitem{richard} Richard, T. 
\textit{Lower bounds on Ricci flow invariant curvatures and geometric applications.} to appear in J. Reine Angew. Math.
{\tt arXiv:1111.0859}

\bibitem{simon} 
Simon, M., 
\textit{Ricci flow of almost non-negatively curved three manifolds.} 
J. Reine Angew. Math. 
2009 (630),
177--217.

\bibitem{simon-non-collapsed}
Simon, M.,
\textit{Ricci flow of non-collapsed three manifolds whose Ricci curvature is bounded from below.}
J. Reine Angew. Math. 662 (2012), 59--94.


\bibitem{spindeler}
Spindeler, W.,
\textit{$S^1$-actions on 4-manifolds and fixed point
homogeneous manifolds of nonnegative
curvature.}
Ph.D. Thesis. 2014.

\end{thebibliography}

\end{document}
